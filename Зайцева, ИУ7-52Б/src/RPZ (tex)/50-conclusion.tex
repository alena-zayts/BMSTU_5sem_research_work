\chapter*{Заключение}
\addcontentsline{toc}{chapter}{Заключение}

В результате выполнения работы были проведены анализ предметной области и обзор существующих методов решения задачи упрощения текстов. Также были рассмотрены автоматические способы оценки качества упрощения, сформулированы критерии выбора подходящего решения и проведена классификация методов.

На основе классификации был сделан выбор в пользу решений, использующих глубокое обучение. Эти методы, во-первых, решают именно задачу упрощения текстов, а не задачи суммаризации или сокращения, во-вторых, наиболее комплексно упрощают текст, учитывая его лексическую и структурную составляющие.

Был сделан вывод, что выбор из методов глубокого обучения оказывает меньшее влияние на конечный результат, чем использование дополнительных показателей для фильтрации обучающих данных или для выбора наиболее подходящего упрощения из созданных, и что выбор оптимальных параметров требует дополнительного исследования.







