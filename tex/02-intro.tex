\chapter*{Введение}
\addcontentsline{toc}{chapter}{Введение}

Целью упрощения текста является его преобразование в более легкую для чтения и понимания форму. Решение этой задачи всегда было необходимо для отдельных групп людей\cite{liu_simplification_2016}\cite{evans_evaluation_2014} и для подготовки данных в других задачах обработки естественного языка\cite{finegan_dollak_sentence_2016}, а ее актуальность возрастает в связи с резким увеличением количества неструктурированных текстовых данных из-за развития Интернета.

В процессе упрощения текста нужно рассматривать сразу несколько его сотставляющих. Например, необходимо делать проще его лексику и структуру, при этом сохраняя смысловую часть неизменной.

Оценка качества упрощения также не является тривиальной задачей. Из-за субъективности такой оценки, несмотря на появление метрик, которые позволяют это делать автоматически, оценка текста несколькими людьми все еще считается наиболее достоверной\cite{shardlow_survey_2014}.

Целью данной работы является выбор метода, который наиболее полно решает задачу упрощения текстов.

В рамках выполнения работы необходимо решить следующие задачи: 
\begin{itemize}
	\item провести анализ предметной области;
	\item рассмотреть существующие метрики оценки качества упрощения;
	\item провести анализ существующих решений задачи упрощения текстов;
	\item сформулировать критерии выбора решения;
	\item на основе этих критериев провести классификацию решений;
	\item определить, какой метод или методы являются лучшими по совокупности критериев.
\end{itemize}
