\chapter{Анализ предметной области}

В данном разделе вводятся основные определения и описываются важность и актуальность задачи упрощения текстов на русском языке.

\section{Конвейерная обработка данных}


Существует различные формулировки задачи упрощения текста. 
Так, в статье (https://www.jbe-platform.com/content/journals/10.1075/itl.165.2.06sid) даются определения в двух смыслах:

Упрощение текста в узком смысле - это процесс уменьшения лингвистической сложности текста при сохранении исходной информации и смысла. 
В более широком смысле упрощение текста охватывает другие операции: смысловое изменение для упрощения как формы, так и содержания; краткое изложение текста для исключения второстепенной или избыточной информации информации.

В статье Muss упрощением предложений называют процесс, целью которого является получение более легкого для чтения и понимания предложения за счет уменьшения его лексической и синтаксической сложности.

При этом задача упрощения текстов относится к области NLP и имеет много общего с другими задачами из этой сферы - машинным переводом, перефразированием и обобщением (резюмированием) текста (Чжу и др., 2010). 

Если использовать более широкое понятие задачи упрощения текста, то ее будет сложно отличить от задачи обобщения. Поэтому, чтобы разграничить эти два понятия, в данной работе будет использоваться более узкое определение задачи упрощения.

В таком случае упрощение отличается от обобщения тем, что во втором случае основное внимание уделяется сокращению длины и содержания исходных данных. И хотя обобщенные тексты, как правило, короче, это не всегда так, и обобщение может привести к увеличению длины полученных предложений (Шардлоу, 2014). В рамках же упрощения текста обычно сохраняется все содержание.



\section{Выводы из аналитиза предметной области}

